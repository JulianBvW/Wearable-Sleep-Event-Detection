\chapter{Conclusion \label{Chapter-Conclusion}}

% Key contribution
In this work, we presented an SDB detection model with state-of-the-art performance, that can address the large number of undiagnosed sleep apnea cases, due to its selection of uncomplicated and inobtrusive input sensors.

% Important results
We achieved an F1-score of 69.7\% on an event detection level and an AHI prediction correlation of $\rho = 0.917$. When diagnosing normal breathing versus SDB, we achieved a positive likelihood ratio of 11.9 and a negative likelihood ratio of 0.018, showing great discriminatory performance.
Using sensors of similar levels of obtrusiveness, other literature shows equal or lower performances. 
While studies with better performance exist, they mainly use airflow measurements, which are harder to set up and can disrupt or alter sleep. We opt for using PPG and SpO2, signals, which are less complex to set up and arguably more comfortable.

% Clinical implications
Making SDB detection methods as easy as possible is important, as SDB and in particular OSA can have severe clinical implications when untreated, like sleepiness, hypertension, fatigue, and cardiovascular diseases. This together with the fact, that of the estimated 900 million people with SDB worldwide, about 90\% are undiagnosed, shows the need for an easy way of detecting SDB. Our model can help with screening, evaluating night-to-night variability in patients, and give insight into the efficiency of apnea treatment methods.

% SpO2 is optional
We also showed a less powerful version of our model that uses only PPG, enabling its use with even less obtrusive measuring hardware like smartwatches and smart rings, which, to our knowledge, cannot reliably measure SpO2 yet.
While event detection performance dropped to F1 = 61.6\% and AHI correlation dropped to $\rho = 0.917$, we still think that this convenient way could help with greatly the underdiagnosis of SDB, when used over multiple nights. 

% Sleep stage importance
Another important finding is the importance of correct sleep stage classification. Our research showed that performance decreases when using less certain sleep stage information. This is due to the fact how SDB events manifest in different stages of sleep. For example, N3 shows protective characteristics against apnea events. While the ground-truth hypnogram resulted in the best apnea detection performance, this approach is not usable in practice, as it came from signals of a full PSG, which is not feasible for large-scale screening and is practically limited to one or two nights.

% Preprocessing differences
We also tried preprocessing the PPG signals, instead of feeding the neural network the raw signal. We tested a statistical approach, that extracted features like mean, max, and peak difference from 5-second windows, and a VAE approach, which learned a latent representation for the PPG signal, that could be used instead. Although training time was reduced significantly by a factor of 3x, we only achieved results reaching the in-model performance by using a combination of statistical and VAE preprocessing. Furthermore, this reduction in training time does not matter in practice, as the preprocessing still needs to be done for inference. Still, this might be a good solution for rapid prototyping.

% Output correction
Finally, we found that correcting the model's output, by disregarding events shorter than 3 seconds and merging events that are closer together than 3 seconds, helped with performance.

% Limitations and future work
Future work is needed to evaluate our model's performance on other, external datasets, as we exclusively used the MESA dataset. This, while being large-scale and including different co-morbidities, still does not represent real-world patient distribution.
Also, further research could look into using other, unobtrusive modalities like snoring sounds for SDB detection and refining AHI prediction by incorporating demographic features.

% Final words
In conclusion, we believe our work is a step in the direction of reliable SDB detection with broad application capabilities. We hope that future research further improves our results and leads the world toward healthy sleep.