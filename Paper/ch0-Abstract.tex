\chapter*{Abstract}

\textbf{Objective}: In this work, we present an automatic detection model for Sleep-Disordered Breathing (SDB) events, particularly apnea events. Sleep Apnea is a common sleep disorder affecting roughly 10\% of the population, and is associated with severe health risks. We aim to develop an easy and unobtrusive way of diagnosing SDB that helps with the large number of undiagnosed cases.

\noindent\textbf{Methods}: We trained an Attention U-Net architecture on PPG and optionally SpO2 signals from 1,880 overnight recordings of the MESA dataset to detect SDB events on a second-by-second basis. We evaluated the model performance using a strict event-level scoring method and compared the results to those of other studies. Another input to the model was the sleep stage information obtained from a pre-trained cardiorespiratory sleep staging model that used only PPG.

\noindent\textbf{Results}: We achieved a peak event-level detection F1-score of 69.7\% and a correlation of $\rho = 0.917$ for AHI prediction using PPG and SpO2. Reducing this setup to only use PPG, we achieved an F1-score of 61.6\% and a correlation of $\rho = 0.842$. We also showed the importance of using accurate sleep stage information, as providing the model with the ground-truth hypnogram increased the F1-score to 76.1\%, while not using any sleep stage information at all resulted in an F1-score of 56.9\%.

\noindent\textbf{Conclusion}: We present a state-of-the-art SDB detection model that uses only PPG and SpO2 signals, which are easy to set up and unobtrusive during sleep. We also show that using surrogate sleep stage information is a valid approach to improve performance, even if it is not as good as the ground-truth hypnogram. Our model can help with the large number of undiagnosed SDB cases and can be used for long-term monitoring of patients.