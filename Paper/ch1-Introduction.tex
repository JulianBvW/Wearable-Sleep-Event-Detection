\chapter{Introduction \label{Chapter-Intro}}

A recent study estimated that over 900 million adults globally are affected
 by the common group of respiratory sleep disorders called Sleep-disordered
 breathing (SDB) [bla-F5], or in particular, Sleep Apnea, which clinical
 manifestations include sleepiness, fatigue, cardiovascular disease, and
 hypertension. SDB is even linked to higher cases of diabetes, stroke
 occurrences and increased morbidity [bla-2,3,4].
Diagnosis of the disorder relies on detecting repeated respiratory events
 in which airflow is either reduced (hypopnea) or entirely paused (apnea)
 during sleep [bla-1,2]. These events can further be categorized into
 obstructive or central origin, depending on if the apnea happens due to
 a physical blockage of the upper airway or if caused by the brain failing
 to signal breathing resulting in missing breathing effort. In case the
 event shows features of both, it is classified as a mixed.
Dividing the number of events by the total sleep time (TST) gives the
 Apnea-Hypopnea-Index (AHI), which indicates the severeness of the disorder.

Gold standard for detecting SDB in Polysomnography (PSG) which captures
 physical and biological signals like heart (electrocardiogram, ECG) and
 brain (electroencephalogram, EGG) activity, airflow, peripheral oxygen
 saturation (SpO2), chest and abdominal movements, sleeping position, and
 blood volume changes (photoplethysmographie, PPG).
This approach comes with a few downsides: Firstly, due to the vast amount
 of sensors and specialized equipment, setup and analysis of the full PSG
 is costly, requires human experts and might impact sleep quality.
 Secondly, looking only at a single night might have low diagnostic
 meaningfulness [bla-5] and the analysis of multiple nights is needed. All
 this contributes to the fact that an estimated 93\% of women and 82\%
 of men with at least moderate SDB are undiagnosed [bla-A5=D4].

In 2000, PhysioNet started interest in the topic of less complex apnea
 detection by holding a competition on their Apnea-ECG Dataset that
 only consists of labeled ECG recordings split into one-minute epochs.
 Although presented models reached high performances, later studies showed
 poor generalizability for these models and indicated that the dataset
 doesn't fully cover the broad spectrum of apneic events [bla-6]. Therefore,
 in the last decades, a wide range of sleep disorder datasets and apnea
 detection architectures were published that focused on generalizability.
For instance, Olsen et al. \cite{olsen2020robust} used bidirectional GRUs
 on ECG data to achieve a sensitivity (Se) of 68.7\%, a precision (Pr)
 of 69.1\%, and an F1-score of 66.6\% on their self-defined event-level
 metric and an AHI-correlation of $R^2$ = 0.829. Xie et al. \cite{xie2023use}
 later validated Olsens model on the SOMNIA dataset and achieved an
 F1-score of 0.708 using the PSG-computed hypnogram and an F1-score of
 0.631 with their Multi-Task model that computed sleep stages based on
 ECG and respiratory effort (RE) only \cite{xie2024multi}. Also using 
 ECG and RE, Fonseca et al. \cite{fonseca2024estimating} achieved intraclass
 correlation coefficient of 0.91 across different datasets.\todo{OTHER, OTHER}

One of the simplest signals to record while sleep is PPG, which can be
 obtained through the use of a pulse oximeter that illuminates the skin
 to measure changes in light absorption.
These devices come in a range of forms such as wrist-worn, like most
 modern smart-watch already have, or finger-worn, which can also
 calculate SpO2.
\todo{OTHER used these sensors, OTHER also}

In this work, we present an event-level apnea detection model that relies
 solely on signals obtained by easy to use recording hardware, namely
 PPG and SpO2, and show the importance of correct sleep stage identification.
