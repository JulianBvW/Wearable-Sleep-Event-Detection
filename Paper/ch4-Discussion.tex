\chapter{Discussion \label{Chapter-Discussion}}

% Discussions

In this work, we presented an automatic SDB detection model based on an Attention U-Net using as input only PPG and optionally SpO2 signals. We achieved a peak F1-score of 69.7\% in event detection and an AHI prediction correlation of $\rho = 0.917$. We showed great diagnostic results with positive and negative likelihood ratios of $\ge 11.0$ and $\le 0.1$ respectively with very few participants ($\le 1\%$) being wrongly classified more than one severity class apart. Event detection metrics were based on a strict event scoring that is more transparent than minute-to-minute, segment classification, as used in other studies.

Our model demonstrated higher detection rates for apnea events compared to hypopnea events. The much larger prevalence of hypopneas than any other event type in our dataset would suggest, that the classifier had more examples to learn from and could yield higher performance for these. However, this result is expected from a physiological perspective because hypopneas, which are associated with partial instead of full obstructions, have a much lower expression in cardiorespiratory changes in the PPG signal, and are not always associated with desaturation events, and thus have no expression in the SpO2 signal.

One goal of our work was to use only PPG and SpO2, as the finger-worn sensor used to record these signals, is easy to set up and relatively unobtrusive during sleep. However, even less obtrusive options are smartwatches or smart rings, which already today can record PPG with good quality. To the best of our knowledge, SpO2 cannot yet be reliably recorded using these devices, especially not the subtle drops in saturation of as little as 3\% based on which respiratory events are scored. 
We showed that omitting SpO2 data from the training signals decreased performance, but not by a drastic amount. The model was still able to detect SDB events with a peak F1-score of 61.6\% and predict AHI with a correlation of $\rho = 0.842$. This means that our model is still useful using these even less obtrusive technologies that only measure PPG, and can be used over many nights. This may enable long-term home monitoring of SDB and accurate screening, as well as help with evaluating night-to-night variability and the effectiveness of SDB treatment methods.

Another important determinant of performance for our model is the source of sleep stage labels. The model performed best when using sleep stages scored with Somnolyzer, and worst without any sleep stage information. The PPG-derived sleep stages from the algorithm described by Bakker et al. \cite{bakker2021estimating} greatly improved results over using no sleep stage information but were still not as good as using the ground-truth hypnogram. As described in Appendix \ref{Apx-Pred-Hypnogram}, the algorithm can reach higher performance when besides cardiac information, also respiratory information is available. Further improvements in predicting the hypnogram from only PPG signals could lead to better results in detecting SDB, that still rely solely on data from PPG sensors.
Although surrogate measures for predicting the hypnogram will likely never reach the same quality as the full PSG-derived sleep stages, our results indicate that even imperfect sleep stage information is well-suited for this task.

We also evaluated several approaches for preprocessing the PPG signal, namely using statistical analysis and a VAE. While achieving the same level of performance as using the in-model approach, the training time of the detection model decreased significantly. Inference time will not be affected, as the benefit comes only from not needing to recompute the preprocessed signal for each training run, but this approach could help with rapid prototyping and hyperparameter tuning.

Finally, we corrected the model output by filtering out events shorter than 3 seconds and merging events less than 3 seconds apart into one, which yielded better results than not correcting the output at all.
Analyzing the precision and recall separately, we found that increasing this correction size beyond 3 seconds, results in great decreases in recall, while the precision just improved slightly, resulting in a worse overall F1-score. The opposite happened when using correction sizes smaller than 3, where the recall improvements could not compensate for the big drop in precision. This big drop is likely due to single outliers for one or two seconds, which we can get rid of by selecting a correction size of 3 seconds.

\section{Limitations and Future Work}

% Limitations

An important limitation of our work is the lack of validation on other, external datasets. While the MESA dataset used in this work is large and greatly balanced in some cohort statistics, like AHI, BMI, smoking habits, or presence of co-morbidities, other factors like age are not balanced. Recordings have also been made in a clinical setting and with the same hardware. Even further, first-night effects were not addressed.
Validation of our work on other datasets is crucial to show the generalizability and usefulness of our model in the real world.

% Future work

Future work could tackle the bias and errors in AHI prediction, especially with the PPG-only model. While a linear correction could help correcting the bias, other studies have shown that using demographic data to refine the AHI through a small MLP can increase correlation greatly.

Also, we showed that training without SpO2 data was, while plateauing way slower, more stable than training with it, which is likely due to the fact that SpO2 in itself is less stable and prone to artifacts. As the F1-score and losses did not seem to reach their peak in the 30 epochs we trained the model for, further work could look into training for longer, or using other preprocessing techniques for the SpO2 signal.

As the use of smartwatches or smart rings maximizes our goal of unobtrusive SDB detection even further, future work could also explore other sensors that are already available on these devices. One example is the accelerometer, which records movements during sleep, or breathing sounds, that can be easily monitored with an associated smartphone, and which can be used for detecting snoring. Both signals are indicators of SDB and might improve detection performance even further.